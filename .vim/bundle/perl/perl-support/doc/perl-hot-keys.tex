%%=====================================================================================
%%
%%         File:  perl-hot-keys.tex
%%
%%  Description:  perl-support.vim : Key mappings for Vim without GUI.
%%                
%%                
%%       Author:  Dr.-Ing. Fritz Mehner
%%        Email:  mehner@fh-swf.de
%%    Copyright:  Copyright (C)  2003-2013  Dr.-Ing. Fritz Mehner (mehner.fritz@fh-swf.de)
%%      Version:  see \Pluginversion
%%      Created:  06.06.2003
%%                
%%=====================================================================================

%%======================================================================
%%  LaTeX settings       [[[1
%%======================================================================

\documentclass[oneside,10pt,landscape,DIV17]{scrartcl}

\usepackage[english]{babel}
\usepackage[utf8]{inputenc}
\usepackage[T1]{fontenc}
\usepackage{lastpage}
\usepackage{multicol}
\usepackage{fancyhdr}

\setlength\parindent{0pt}

\newcommand{\Pluginversion}{5.2}
\newcommand{\ReleaseDate}{\today}
\newcommand{\Rep}{{\scriptsize{[n]}}}

%%----------------------------------------------------------------------
%%  fancyhdr
%%----------------------------------------------------------------------
\pagestyle{fancyplain}
\fancyhf{}
\fancyfoot[L]{\small \ReleaseDate}
\fancyfoot[C]{\small perl-support.vim}
\fancyfoot[R]{\small \textbf{Page \thepage{} / \pageref{LastPage}}}
\renewcommand{\headrulewidth}{0.0pt}

%%----------------------------------------------------------------------
%%  luximono : Type1-font
%%  Makes keyword stand out by using semibold letters.
%%----------------------------------------------------------------------
\usepackage[scaled]{luximono}

%%----------------------------------------------------------------------
%%  hyperref
%%----------------------------------------------------------------------
\usepackage[ps2pdf]{hyperref}
\hypersetup{pdfauthor={Dr.-Ing. Fritz Mehner, FH Südwestfalen, Iserlohn, Germany}}
\hypersetup{pdfkeywords={Vim, Perl}}
\hypersetup{pdfsubject={Vim-plug-in,  perl-support.vim, hot keys}}
\hypersetup{pdftitle={Vim-plug-in,  perl-support.vim, hot keys}}


%%%%%%%%%%%%%%%%%%%%%%%%%%%%%%%%%%%%%%%%%%%%%%%%%%%%%%%%%%%%%%%%%%%%%%%%
%%  START OF DOCUMENT
%%%%%%%%%%%%%%%%%%%%%%%%%%%%%%%%%%%%%%%%%%%%%%%%%%%%%%%%%%%%%%%%%%%%%%%%
\begin{document}%

\begin{multicols}{3}
%
\begin{center}
%
%%======================================================================
%%  title				[[[1
%%======================================================================
\textbf{\textsc{\small{Vim-Plug-in}}}\\
\textbf{\LARGE{perl-support.vim}}\\
\textbf{\textsc{\small{Version \Pluginversion}}}\\
\vspace{5mm}%
\textbf{\textsc{\Huge{Hot keys}}}\\ 
\vspace{5mm}%
Key mappings for Vim and gVim.\\
Plug-in: http://vim.sourceforge.net\\
Fritz Mehner (mehner.fritz@fh-swf.de)\\
\vspace{1.0mm}
{\normalsize (i)} insert mode, {\normalsize (n)} normal mode, {\normalsize (v)} visual mode\\
\vspace{4.0mm}

%%======================================================================
%%  page 1, table, left part				[[[1
%%======================================================================
%%~~~~~ TABULAR : begin ~~~~~~~~~~
\begin{tabular}[]{|p{11mm}|p{60mm}|}
%%----------------------------------------------------------------------
%%  main menu
%%----------------------------------------------------------------------
\hline 
\multicolumn{2}{|r|}{\textsl{\textbf{P}erl}}\\[1.0ex]
\hline \verb'\ft'    & file tests                      \hfill (n,i)\\
\hline 
%%----------------------------------------------------------------------
%%  show plug-in help
%%----------------------------------------------------------------------
\hline 
\multicolumn{2}{|r|}{\textsl{\textbf{H}elp}}\\[1.0ex]
\hline \verb'\h'    & read perldoc for word under cursor \hfill (n,i)\\
\hline \verb'\hp'   & help (plug-in) \hfill (n,i)\\
\hline 
%%----------------------------------------------------------------------
%%  menu comments
%%----------------------------------------------------------------------
\hline
\multicolumn{2}{|r|}{\textsl{\textbf{C}omments}}                       \\[1.0ex]
\hline \Rep\verb'\cl'   & end-of-line comment               \hfill (n, v, i)\\
\hline \Rep\verb'\cj'   & adjust end-of-line comments       \hfill (n, v, i)\\
\hline     \verb'\cs'   & set end-of-line comment col.      \hfill (n)      \\
%
\hline \Rep\verb'\cc'   & code $\leftrightarrow$ comment    \hfill (n, v)   \\
\hline     \verb'\cb'   & code block $\rightarrow$ comment  \hfill (n, v)   \\
\hline     \verb'\cub'  & uncomment code block              \hfill (n)      \\
%
\hline     \verb'\cfr'  & frame comment                     \hfill (n, i)   \\
\hline     \verb'\cfu'  & function description              \hfill (n, i)   \\
\hline     \verb'\cme'  & method description                \hfill (n, i)   \\
\hline     \verb'\chpl' & file header (.pl)                 \hfill (n)      \\
\hline     \verb'\chpm' & file header (.pm)                 \hfill (n)      \\
\hline     \verb'\cht'  & file header (.t)                  \hfill (n)      \\
\hline     \verb'\chpo' & file header (.pod)                \hfill (n)      \\
\hline     \verb'\cd'   & date                              \hfill (n, i)   \\
\hline     \verb'\ct'   & date \& time                      \hfill (n, i)   \\
\hline     \verb'\ck'   & keyword comments                  \hfill (n, i)   \\
\hline     \verb'\cma'  & plug-in macros                     \hfill (n, i)   \\
\hline
\end{tabular}\\
%%~~~~~ TABULAR :  end  ~~~~~~~~~~
%
%%======================================================================
%%  page 1, table, middle part				[[[1
%%======================================================================
%
%%~~~~~ TABULAR : begin ~~~~~~~~~~
\begin{tabular}[]{|p{11mm}|p{60mm}|}
%%----------------------------------------------------------------------
%%  menu statements
%%----------------------------------------------------------------------
\hline
\multicolumn{2}{|r|}{\textsl{\textbf{S}tatements}}                    \\[1.0ex]
\hline \verb'\sd'      & \verb'do { } while'          \hfill (n, v, i)\\
\hline \verb'\sf'      & \verb'for { }'               \hfill (n, v, i)\\
\hline \verb'\sfe'     & \verb'foreach { }'           \hfill (n, v, i)\\
\hline \verb'\si'      & \verb'if { }'                \hfill (n, v, i)\\
\hline \verb'\sie'     & \verb'if { } else { }'       \hfill (n, v, i)\\
\hline \verb'\se'      & \verb'else { }'              \hfill (n, v, i)\\
\hline \verb'\sei'     & \verb'elsif { }'             \hfill (n, v, i)\\
\hline \verb'\su'      & \verb'unless { }'            \hfill (n, v, i)\\
\hline \verb'\sue'     & \verb'unless { } else { }'   \hfill (n, v, i)\\
\hline \verb'\st'      & \verb'until { }'             \hfill (n, v, i)\\
\hline \verb'\sw'      & \verb'while { }'             \hfill (n, v, i)\\
\hline \verb'\sg'      & \verb'given { }'             \hfill (n, v, i)\\
\hline \verb'\swh'     & \verb'when { }'              \hfill (n, v, i)\\
\hline
%%----------------------------------------------------------------------
%%  menu idioms
%%----------------------------------------------------------------------
\hline
\multicolumn{2}{|r|}{\textsl{\textbf{I}dioms}}                 \\[1.0ex]
\hline \verb'\id  '  & \verb'my $;'              \hfill (n, i)   \\
\hline \verb'\ida '  & \verb'my $ = ;'           \hfill (n, i)   \\
\hline \verb'\idd '  & \verb'my ( $, $ );'       \hfill (n, i)   \\
\hline \verb'\ia  '  & \verb'my @;'              \hfill (n, i)   \\
\hline \verb'\iaa '  & \verb'my @ = (,,);'       \hfill (n, i)   \\
\hline \verb'\ih  '  & \verb'my %;'              \hfill (n, i)   \\
\hline \verb'\iha '  & \verb'my % = (=>,=>,);'   \hfill (n, i)   \\
%
\hline \verb'\ir'  & \verb'my $rgx_ = q//;'    \hfill (n, i)   \\
\hline \verb'\im'  & \verb'$ =~ m//xm'         \hfill (n, i)   \\
\hline \verb'\is'  & \verb'$ =~ s///xm'        \hfill (n, i)   \\
\hline \verb'\it'  & \verb'$ =~ tr///xm'       \hfill (n, i)   \\
\hline \verb'\isu' & \verb'subroutine'         \hfill (n, v, i)\\
\hline \verb'\ip'  & \verb'print "...\n";'     \hfill (n ,i)   \\
\hline \verb'\ii'  & open input file           \hfill (n, v, i)\\
\hline \verb'\io'  & open output file          \hfill (n, v, i)\\
\hline \verb'\ipi' & open pipe                 \hfill (n, v, i)\\
\hline
%%----------------------------------------------------------------------
%%  snippet menu
%%----------------------------------------------------------------------
\hline
\multicolumn{2}{|r|}{\textsl{S\textbf{n}ippet}}                \\[1.0ex]
\hline \verb'\nr'  & read code snippet         \hfill (n, i)   \\
\hline \verb'\nv'  & view code snippet         \hfill (n, i)   \\
\hline \verb'\nw'  & write code snippet        \hfill (n, v, i)\\
\hline \verb'\ne'  & edit code snippet         \hfill (n, i)   \\
%
\hline \verb'\ntl' & edit local templates      \hfill (n, i)\\
\hline \verb'\ntr' & reread the templates      \hfill (n, i)\\
\hline \verb'\nts' & choose template style     \hfill (n, i)\\
%
\hline \verb'\njt' & insert jump tag           \hfill (n, i)\\
\hline \verb'\nxs' & regex snippet template    \hfill (n, i)\\
\hline
\end{tabular}\\
%%~~~~~ TABULAR :  end  ~~~~~~~~~~
%
%%======================================================================
%%  page 1, table, right part				[[[1
%%======================================================================
%
%%~~~~~ TABULAR : begin ~~~~~~~~~~
\begin{tabular}[]{|p{11mm}|p{60mm}|}
%%----------------------------------------------------------------------
%%  menu regex menu
%%----------------------------------------------------------------------
\hline
\multicolumn{2}{|r|}{\textsl{Regular E\textbf{x}pressions}}     \\[1.0ex]
\hline     \verb'xpc' &  POSIX classes                 \hfill (n, i)\\ 
\hline     \verb'xup' &  Unicode properties            \hfill (n, i)\\ 
\hline     \verb'xex' &  extended Regex                \hfill (n, i)\\ 
\hline     \verb'xms' &  metasymbols                   \hfill (n, i)\\ 
%
\hline \Rep\verb'\xr' &  pick up Regex                 \hfill (n, v)\\
\hline \Rep\verb'\xs' &  pick up string                \hfill (n, v)\\
\hline     \verb'\xf' &  pick up flag(s)               \hfill (n, v)\\
\hline     \verb'\xm' &  match                         \hfill (n)   \\
\hline     \verb'\xmm'&  match multiple (Regex/target) \hfill (n)   \\
\hline     \verb'\xe' &  explain Regex                 \hfill (n, v)\\
\hline
%%----------------------------------------------------------------------
%%  menu Special variables
%%----------------------------------------------------------------------
\hline
\multicolumn{2}{|r|}{\textsl{\textbf{S}pecial Variables}}            \\[1.0ex]
\hline \verb'\vb'   & basics                           \hfill (n, i)\\
\hline \verb'\ve'   & errors                           \hfill (n, i)\\
\hline \verb'\vf'   & files                            \hfill (n, i)\\
\hline \verb'\vid'  & IDs                              \hfill (n, i)\\
\hline \verb'\vio'  & IO                               \hfill (n, i)\\
\hline \verb'\vr'   & regexp                           \hfill (n, i)\\
\hline \verb'\vs'   & POSIX signals                    \hfill (n, i)\\
\hline \verb'\vue'  & use English                      \hfill (n, i)\\
\hline
%%----------------------------------------------------------------------
%%  menu POD
%%----------------------------------------------------------------------
\hline
\multicolumn{2}{|r|}{\textsl{\textbf{P}OD}}                       \\[1.0ex]
\hline \verb'\ppc'    & \verb'pod, cut'                 \hfill (n, i)\\
\hline \verb'\pfc'    & \verb'for, cut'                 \hfill (n, i)\\
\hline \verb'\pbh'    & \verb'begin html, end'          \hfill (n, i)\\
\hline \verb'\pbm'    & \verb'begin man, end'           \hfill (n, i)\\
\hline \verb'\pbt'    & \verb'begin text, end'          \hfill (n, i)\\
\hline \verb'\ph1'    & \verb'head1'                    \hfill (n, i)\\
\hline \verb'\ph2'    & \verb'head2'                    \hfill (n, i)\\
\hline \verb'\ph3'    & \verb'head3'                    \hfill (n, i)\\
\hline \verb'\pob'    & \verb'over, back'               \hfill (n, i)\\
\hline \verb'\pi'     & \verb'item'                     \hfill (n, i)\\
\hline \verb'\pod'    & run \verb'podchecker'           \hfill (n, i)\\
\hline \verb'\podh'   & convert POD data to .html file  \hfill (n, i)\\
\hline \verb'\podm'   & Convert POD data to *roff input \hfill (n, i)\\
\hline \verb'\podt'   & Convert POD data to ASCII text  \hfill (n, i)\\
\hline \verb'\pm'     & markup sequences                \hfill (n, i)\\
\hline
\end{tabular}\\
%%~~~~~ TABULAR :  end  ~~~~~~~~~~
%
%
\end{center}%
\end{multicols}%
%
\newpage
%
%%======================================================================
%%  page 2, table, left part				[[[1
%%======================================================================
%
\begin{multicols}{2}
%
%%~~~~~ TABULAR : begin ~~~~~~~~~~
\begin{tabular}[]{|p{11mm}|p{61mm}|}
%%----------------------------------------------------------------------
%%  menu Profiling
%%----------------------------------------------------------------------
\hline
\multicolumn{2}{|r|}{\textsl{\textbf{P}rofiling}}                 \\[1.0ex]
\hline \verb'\rps'     & run \verb'SmallProf'                    \hfill (n, i)\\
\hline \verb'\rpss'    & sort \verb'SmallProf' report            \hfill (n, i)\\
\hline \verb'\rpso'    & open existing \verb'SmallProf' results  \hfill (n, i)\\
\hline
%
\hline \verb'\rpf'     & run \verb'FastProf'                    \hfill (n, i)\\
\hline \verb'\rpfs'    & sort \verb'FastProf' report            \hfill (n, i)\\
\hline \verb'\rpfo'    & open existing \verb'FastProf' results  \hfill (n, i)\\
%
\hline
\hline \verb'\rpn'     & run \verb'NYTProf'                    \hfill (n, i)\\
\hline \verb'\rpns'    & sort \verb'NYTProf' report            \hfill (n, i)\\
\hline \verb'\rpno'    & open existing \verb'NYTProf' results  \hfill (n, i)\\
\hline \verb'\rpnh'    & browse HTML files (\verb'NYTProf')\hfill (n, i)\\
\hline
%%----------------------------------------------------------------------
%%  menu run
%%----------------------------------------------------------------------
\hline
\multicolumn{2}{|r|}{\textsl{\textbf{R}un}} \\[1.0ex]
\hline \verb'\rr'    & update file, run script                  \hfill (n, i)   \\
\hline \verb'\rs'    & update file, check syntax                \hfill (n, i)   \\
\hline \verb'\ra'    & set command line arguments               \hfill (n, i)   \\
\hline \verb'\rw'    & set Perl cmd.\ line switches             \hfill (n, i)   \\
\hline \verb'\re'    & make script executable                   \hfill (n, i)   \\
\hline \verb'\rd'    & start debugger                           \hfill (n, i)   \\
\hline \verb'\ri'    & show installed Perl modules              \hfill (n, i)   \\
\hline \verb'\rg'    & generate Perl module list                \hfill (n, i)   \\
\hline \verb'\ry'    & run \verb'perltidy'                      \hfill (n, v, i)\\
\hline \verb'\rpc'   & run \verb'perlcritic'                    \hfill (n, i)   \\
\hline \verb'\rpcs'  & set \verb'perlcritic' severity           \hfill (n, i)   \\
\hline \verb'\rpcv'  & set \verb'perlcritic' verbosity          \hfill (n, i)   \\
\hline \verb'\rpco'  & set \verb'perlcritic' options            \hfill (n, i)   \\
\hline \verb'\rt'    & save buffer with timestamp               \hfill (n, i)   \\
\hline \verb'\rh'    & hardcopy buffer                          \hfill (n, v, i)\\
\hline \verb'\rk'    & settings and hotkeys                     \hfill (n, i)   \\
\hline \verb'\rx'    & set xterm size                           \hfill (n, i {\tiny GUI only})\\
\hline \verb'\ro'    & change output destination                \hfill (n, i)   \\
\hline
%%----------------------------------------------------------------------
%%  menu run
%%----------------------------------------------------------------------
\hline
\multicolumn{2}{|r|}{\textsl{Tool Box}} \\[1.0ex]
\hline \verb'\rm'    & run \texttt{make}                        \hfill (n, i)   \\
\hline \verb'\rcm'   & choose makefile                          \hfill (n, i)   \\
\hline \verb'\rmc'   & \texttt{make clean}                      \hfill (n, i)   \\
\hline \verb'\rma'   & command line arguments for \texttt{make} \hfill (n, i)   \\
\hline
\end{tabular}%
%%~~~~~ TABULAR :  end  ~~~~~~~~~~
%

\parbox[t][70mm][t]{120mm}{%
%
\begin{tabbing}
\hspace{30mm} \= \hspace{50mm} \= \kill
%
%%======================================================================
%%  page 2, table, right part				[[[1
%%======================================================================
%
%%----------------------------------------------------------------------
%%  Run
%%----------------------------------------------------------------------
\large{\textbf{Run}}\\[1.0ex]
%
Specify command line arguments for the script in the current buffer.\\
Use tab expansion to choose a file or a directory.\\[1.0ex]
%
\texttt{ :PerlScriptArguments}  \> \\[1.0ex]
%
Specify command line switches for the Perl interpreter.\\[1.0ex]
%
\texttt{ :PerlSwitches}  \> \\[2.5ex]
%
%
%%----------------------------------------------------------------------
%%  perlcritic
%%----------------------------------------------------------------------
\large{\textbf{Perlcritic}}\\[1.0ex]
%
Ex commands for \texttt{perlcritic} (version 1.01+)\\
Use tab expansion to choose the severity or the verbosity.\\[2.0ex]
\texttt{ :CriticSeverity}  \> \texttt{\ 1\ \ \ \ \ \ 2\ \ \ \ \ 3\ \ \ \ \ 4\ \ \ \ \ 5} \\
                           \> \texttt{\ brutal cruel harsh stern gentle} \\[1.0ex]
\texttt{ :CriticVerbosity} \> \texttt{\ 1} $\ldots$ \texttt{11}\\[1.0ex]
\texttt{ :CriticOptions}   \> option(s), see \texttt{perlcritic(1)}\\[2.5ex]
%
\hspace{40mm} \= \hspace{50mm} \= \kill
%%----------------------------------------------------------------------
%%  regex tester
%%----------------------------------------------------------------------
\large{\textbf{Regular Expression Tester}}\\[1.0ex]
%
Ex command for the regular expression tester. Set control character\\
replacements for newline and tabulator used to display the results\\
of a match, e.g.:\\[1.0ex]

\texttt{ :RegexSubstitutions}   \> \texttt{'\$}\texttt{\~}\texttt{'}  \\[2.5ex]
%
\hspace{30mm} \= \hspace{50mm} \= \kill
%
%%----------------------------------------------------------------------
%%  Profiling
%%----------------------------------------------------------------------
\large{\textbf{Profiling}}\\[1.0ex]
%
The following ex commands can be used to sort a profiler report \\in the quickfix window.\\
Use tab expansion to choose the sort criterion or the file name.\\[2.0ex]
%
For \texttt{Devel::SmallProf}\\[1.0ex]
\texttt{ :SmallProfSort}   \> \texttt{file-name|line-number|line-count|time|ctime}\\[1.0ex]
%
%
For \texttt{Devel::FastProf}\\[1.0ex]
\texttt{ :FastProfSort}    \> \texttt{file-name|line-number|time|line-count}\\[1.0ex]
%
%
For \texttt{Devel::NYTProf}\\[1.0ex]
\texttt{ :NYTProfCSV}      \> Read a CSV-file.\\[1.0ex]
%
\texttt{ :NYTProfHTML}      \> Read the HTML-reports with an external viewer (GUI only).\\[1.0ex]
%
%
\texttt{ :NYTProfSort}     \> \texttt{file-name|line-number|time|calls|time-call}\\
%
\end{tabbing}
}
\end{multicols}%
%
%%----- TABBING :  end  ----------
\end{document}
% vim: foldmethod=marker foldmarker=[[[,]]]
